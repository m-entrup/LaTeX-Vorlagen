% Die Folgenden Pakete sind schon eingebunden (siehe 00_Protokoll.tex):
%\usepackage[utf8]{inputenc}                                                  % Legt die Zeichenkodierung fest, z.B UTF8
%\usepackage[ngerman,english]{babel}                                           % Silbentrennung nach neuer deutscher und englischer Rechtschreibung
%\usepackage{amsmath}                                                          % Mathepaket
%\usepackage{xifthen}                                                          % Wird benötigt um \ifthenelse zu benutzen
%\usepackage[pdftex]{graphicx}                                                 % Zum flexiblen Einbinden von Grafiken, pdftex ist optional
%\usepackage{microtype}                                                        % Abstände zwischen Buchstaben und Wörtern werden verbessert. Dies führt z.B. dazu, dass weniger Silbentrennung genutzt wird.
%\usepackage{units}                                                            % Ermöglicht die Nutzung von \unit[Zahl]{Einheit}
%\usepackage{setspace}                                                         % Einfaches wechseln zwischen unterschiedlichen Zeilenabständen
%\usepackage[pdfpagelabels]{hyperref}                                          % Verlinkt Textstellen im PDF Dokument
%\usepackage[font=small,labelfont=bf,labelsep=endash,format=plain]{caption}    % Darstellung für Caption s.u.
%\usepackage{cite}                                                             % Zusatzfunktionen zum zitieren
%\usepackage{scrpage2}                                                         % Wird für Kopf- und Fußzeile benötigt
%\usepackage{array,dcolumn}                                                    % Beide Pakete werden für die Ausrichtung der Tabellenspalten benötigt
